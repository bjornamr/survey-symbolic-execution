% v2-acmsmall-sample.tex, dated March 6 2012
% This is a sample file for ACM small trim journals
%
% Compilation using 'acmsmall.cls' - version 1.3 (March 2012), Aptara Inc.
% (c) 2010 Association for Computing Machinery (ACM)
%
% Questions/Suggestions/Feedback should be addressed to => "acmtexsupport@aptaracorp.com".
% Users can also go through the FAQs available on the journal's submission webpage.
%
% Steps to compile: latex, bibtex, latex latex
%
% For tracking purposes => this is v1.3 - March 2012

\documentclass[prodmode,acmcsur]{acmsmall} % Aptara syntax

% Package to generate and customize Algorithm as per ACM style
\usepackage[ruled]{algorithm2e} 
\renewcommand{\algorithmcfname}{ALGORITHM}
\SetAlFnt{\small}
\SetAlCapFnt{\small}
\SetAlCapNameFnt{\small}
\SetAlCapHSkip{0pt}
\IncMargin{-\parindent}

% Metadata Information
\acmVolume{0}
\acmNumber{0}
\acmArticle{0}
\acmYear{0000}
\acmMonth{0}

% Copyright
%\setcopyright{acmcopyright}
%\setcopyright{acmlicensed}
%\setcopyright{rightsretained}
%\setcopyright{usgov}
%\setcopyright{usgovmixed}
%\setcopyright{cagov}
%\setcopyright{cagovmixed}

% !TEX root = main.tex
\usepackage{a4wide}
\usepackage{listings}
\usepackage{comment}
\usepackage{amsmath}
\usepackage{graphicx}
\usepackage{amssymb}
\usepackage{url}
\usepackage{hyperref}
\usepackage{float}
\usepackage{lipsum}
\usepackage{caption}
\usepackage{subcaption}
\usepackage{adjustbox}
\usepackage{framed}
\usepackage{multirow}
\usepackage{framed}
\usepackage{enumitem}
\usepackage{epigraph}
\usepackage{wasysym} % \brokenvert
\usepackage{wrapfig}

\usepackage[usenames, dvipsnames]{xcolor}

% commands
%\newcommand{\fullver}{}
\ifdefined\fullver
\newcommand{\iffullver}[2]{#1}
\else
\newcommand{\iffullver}[2]{#2}
\fi

\usepackage{tikz}
\newcommand*\circled[1]{\tikz[baseline=(char.base)]{
            \node[shape=circle,draw,inner sep=2pt] (char) {#1};}}

%\usepackage{titlesec}
 %\titlespacing{\section}{0pt}{*1.2}{*1.2}
 %\titlespacing{\subsection}{0pt}{*1.1}{*1.1}
 %\titlespacing{\subsubsection}{0pt}{*.6}{*.6}
%\titlespacing{\paragraph}{0pt}{*.6}{*.60}
%\titleformat{\paragraph}[runin]{\normalsize\bfseries\scshape}{}{}{}
%Get rid of some extra whitespace in the bibliography
 %\setlength{\bibsep}{0.75pt}
%Get rid of some extra whitespace around float (containing figures)
 %\setlength{\textfloatsep}{4pt plus 0.25pt minus 1pt}
 %\setlength{\intextsep}{4.0pt plus 0.25pt minus .5pt}
 %\setlength{\floatsep}{2pt plus 2pt minus 1pt}
 %\setlength{\abovecaptionskip}{5pt plus 1pt minus 1pt}
% \setlength{\belowcaptionskip}{10pt plus 1pt minus 1pt}
 %\setlength{\parskip}{0pt}

\renewcommand{\epigraphsize}{\footnotesize}
\setlength{\epigraphwidth}{10cm}
%\renewcommand{\epigraphrule}{0pt}

\definecolor{shadecolor}{rgb}{0.92,0.92,0.92}

\hypersetup{
  colorlinks = true, % colours links instead of ugly boxes
  urlcolor = blue, %  colour for external hyperlinks
  linkcolor = black, % colour of internal links
  citecolor = black, % colour of citations
  pdftitle = {A Survey of Symbolic Execution Techniques},
  pdfauthor= {Roberto Baldoni, Emilio Coppa, Daniele Cono D'Elia, Camil Demetrescu, Irene Finocchi}
}	

%\usepackage{xcolor}
%\newcommand{\myedit}[1]{{\leavevmode\color{red}#1}}
%\newcommand{\mytempedit}[1]{{\leavevmode\color{blue}#1}}
%\newcommand{\myedit}[1]{{\color{red}\underline{#1}}}
%\newcommand{\mytempedit}[1]{{\color{black}#1}}
\newcommand{\mytempedit}[1]{\ignorespaces#1}
\newcommand{\revedit}[1]{{\color{blue}#1}}

%\newcommand{\mytempedit}[1]{{\color{blue}#1}}

%\newcommand{\mytempedit}[1]{{\color{blue}\fontfamily{lmdh}\selectfont #1}}

%\setlength{\parindent}{0pt}
\setlength{\FrameSep}{2pt}
\newcommand{\myparagraph}[1]{\medskip\noindent{\bf\small #1.} }
\newcommand{\myparagraphnoperiod}[1]{\medskip\noindent{\bf\small #1} }

% EDIT TO ENABLE NOTES
\newcommand{\mynote}[1]{\ignorespaces}
%\newcommand{\mynote}[1]{\marginpar{\raggedleft{\fontfamily{pbk}\selectfont\scriptsize{\em #1}}}}

\newcommand{\stwoe}{\text{S\textsuperscript{2}E}}
\newcommand{\myinput}[1]{\ifdefined\internalrep \input{../#1} \else \input{#1} \fi}
\newcommand{\missing}{\textbf{XXX}}
%\newcommand{\boxedexample}[1]{\vspace{2mm}\noindent\fbox{\parbox{0.98\textwidth}{{\em Example.} #1}}}

\ifdefined\arxivver
\newcommand{\boxedexample}[1]{
\begin{shaded}
\noindent{\bf\small Example.} #1
\end{shaded}
}
\else
\newcommand{\boxedexample}[1]{
%\vspace{-2mm}
\begin{shaded*}
\noindent{\bf\small Example.} #1
\end{shaded*}
%\vspace{-2mm}
}
\fi




% DOI
\doi{0000001.0000001}

%ISSN
\issn{1234-56789}

% Document starts
\begin{document}

% Page heads
\markboth{R. Baldoni, E. Coppa, D. C. D'Elia, C. Demetrescu, and I. Finocchi}{A Survey of Symbolic Execution Techniques}

% Title portion
\title{A Survey of Symbolic Execution Techniques\\}
\author{ROBERTO BALDONI
\affil{\href{http://www.cis.uniroma1.it/}{Cyber Intelligence and Information Security Research Center}, Sapienza}
EMILIO COPPA
\affil{\href{http://season-lab.github.io}{SEASON Lab}, Sapienza University of Rome}
DANIELE CONO D'ELIA
\affil{\href{http://season-lab.github.io}{SEASON Lab}, Sapienza University of Rome}
CAMIL DEMETRESCU
\affil{\href{http://season-lab.github.io}{SEASON Lab}, Sapienza University of Rome}
IRENE FINOCCHI
\affil{\href{http://season-lab.github.io}{SEASON Lab}, Sapienza University of Rome}
}
% NOTE! Affiliations placed here should be for the institution where the
%       BULK of the research was done. If the author has gone to a new
%       institution, before publication, the (above) affiliation should NOT be changed.
%       The authors 'current' address may be given in the "Author's addresses:" block (below).
%       So for example, Mr. Abdelzaher, the bulk of the research was done at UIUC, and he is
%       currently affiliated with NASA.

\begin{abstract}
Many security and software testing applications require checking whether certain properties of a program hold for any possible usage scenario. For instance, a tool for identifying software vulnerabilities may need to rule out the existence of any backdoor to bypass a program's authentication. One approach would be to test the program using different, possibly random inputs. As the backdoor may only be hit for very specific program workloads, automated exploration of the space of possible inputs is of the essence. Symbolic execution provides an elegant solution to the problem, by systematically exploring many possible execution paths at the same time without necessarily requiring concrete inputs. Rather than taking on fully specified input values, the technique abstractly represents them as symbols, resorting to constraint solvers to construct actual instances that would cause property violations. Symbolic execution has been incubated in dozens of tools developed over the last four decades, leading to major practical breakthroughs in a number of prominent software reliability applications. The goal of this survey is to provide an overview of the main ideas, challenges, and solutions developed in the area, distilling them for a broad audience.
\end{abstract}

%\begin{comment}
\begin{CCSXML} % http://dl.acm.org/ccs.cfm
<ccs2012>
<concept>
<concept_id>10011007.10010940.10010992.10010998.10010999</concept_id>
<concept_desc>Software and its engineering~Software verification</concept_desc>
<concept_significance>500</concept_significance>
</concept>
<concept>
<concept_id>10011007.10010940.10010992.10010998.10011001</concept_id>
<concept_desc>Software and its engineering~Dynamic analysis</concept_desc>
<concept_significance>300</concept_significance>
</concept>
<concept>
<concept_id>10011007.10011074.10011099.10011102.10011103</concept_id>
<concept_desc>Software and its engineering~Software testing and debugging</concept_desc>
<concept_significance>300</concept_significance>
</concept>
<concept>
<concept_id>10002978.10003022</concept_id>
<concept_desc>Security and privacy~Software and application security</concept_desc>
<concept_significance>100</concept_significance>
</concept>
</ccs2012>
\end{CCSXML}

\ccsdesc[500]{Software and its engineering~Software verification}
%\ccsdesc[300]{Software and its engineering~Dynamic analysis}
\ccsdesc[300]{Software and its engineering~Software testing and debugging}
\ccsdesc[100]{Security and privacy~Software and application security}
%\end{comment}

% We no longer use \terms command
%\terms{Design, Algorithms, Performance}

\keywords{Symbolic execution, static analysis, concolic execution, software testing}

\acmformat{Roberto Baldoni, Emilio Coppa, Daniele Cono D'Elia, Camil Demetrescu,
and Irene Finocchi, 2016. A survey of symbolic execution techniques.}
% At a minimum you need to supply the author names, year and a title.
% IMPORTANT:
% Full first names whenever they are known, surname last, followed by a period.
% In the case of two authors, 'and' is placed between them.
% In the case of three or more authors, the serial comma is used, that is, all author names
% except the last one but including the penultimate author's name are followed by a comma,
% and then 'and' is placed before the final author's name.
% If only first and middle initials are known, then each initial
% is followed by a period and they are separated by a space.
% The remaining information (journal title, volume, article number, date, etc.) is 'auto-generated'.

\begin{bottomstuff}
%This work is supported by the National Science Foundation, under grant CNS-0435060, grant CCR-0325197 and grant EN-CS-0329609.

Author's addresses: R. Baldoni, E. Coppa, D.C. D'Elia, and C. Demetrescu, Department of Computer, Control, and Management Engineering, Sapienza University of Rome; I. Finocchi, Department of Computer Science, Sapienza University of Rome. 
This work is supported in part by a grant of the Italian Presidency of the Council of Ministers and by the CINI National Laboratory of Cyber Security. % (Consorzio Interuniversitario Nazionale Informatica) 
\end{bottomstuff}

\maketitle

% !TEX root = main.tex

\epigraph{\textit{``Sometimes you can't see how important something is in its moment, even if it seems kind of important. This is probably one of those times.''}}{(Cyber Grand Challenge highlights from DEF CON 24, August 6, 2016)}

%\vspace{-2.5mm}
\section{Introduction}
\label{se:intro}

Symbolic execution is a popular program analysis technique introduced in the mid '70s to test whether certain properties can be violated by a piece of software~\cite{K-ICRS75,SELECT-ICRS75,K-CACM76,H-TSE77}. Aspects of interest could be that no division by zero is ever performed, no {\tt NULL} pointer is ever dereferenced, no backdoor exists that can bypass authentication, etc. While in general there is no automated way to decide some properties (e.g., the target of an indirect jump), heuristics and approximate analyses can prove useful in practice in a variety of settings, including mission-critical and security applications.

%While in general there is no automated way to decide some properties (think, e.g., of the halting problem), decidable approximations often exist (e.g., ``does a program always terminate within a certain amount of time?''). Such approximations can prove useful in practice in a variety of settings, including mission-critical and security applications.

In a concrete execution, a program is run on a specific input and a single control flow path is explored. Hence, in most cases concrete executions can only under-approximate the analysis of the property of interest. In contrast, symbolic execution can simultaneously explore multiple paths that a program could take under different inputs. This paves the road to sound analyses that can yield strong guarantees on the checked property. 
%\mynote{I: a cosa serve ridirlo? Abbiamo gia' fatto esempi di proprieta' che possono essere verificate}Symbolic execution may answer useful questions on concrete programs like: ``does function {\tt foo(x)} always return a positive value for any possible value of {\tt x}?'' 
The key idea is to allow a program to take on {\em symbolic} -- rather than concrete -- input values. Execution is performed by a {\em symbolic execution engine}, which maintains for each explored control flow path: (i) a first-order Boolean {\em formula} that describes the conditions satisfied by the branches taken along that path, and (ii) a {\em symbolic memory store} that maps variables to symbolic expressions or values. Branch execution updates the formula, while assignments update the symbolic store. A {\em model checker}, typically based on a {\em satisfiability modulo theories} (SMT) solver~\cite{BKM14}, is eventually used to verify whether there are any violations of the property along each explored path and if the path itself is realizable, i.e., if its formula can be satisfied by some assignment of concrete values to the program's symbolic arguments.
%HandbookOfSAT2009

%Variables and control flow paths are associated with expressions and constraints in terms of those symbols during a symbolic execution of the program, and constraints are eventually solved via SMT (satisfiability modulo theories) solvers.

Symbolic execution techniques have been brought to the attention of a heterogeneous audience since DARPA announced in 2013 the Cyber Grand Challenge, a two-year competition seeking to create automatic systems for vulnerability detection, exploitation, and patching in near real-time~\cite{ANGR-SSP16}.
%
% other static program
% which were missed by other program analyses and blackbox testing techniques
More remarkably, symbolic execution tools have been running 24/7 in the testing process of many Microsoft applications since 2008, revealing for instance nearly 30\% of all the bugs discovered by file fuzzing during the development of Windows 7, which other program analyses and blackbox testing techniques missed~\cite{SAGE-QUEUE12}.

In this article, we survey the main aspects of symbolic execution and discuss the most prominent techniques employed for instance in software testing and computer security applications. Our discussion is mainly focused on {\em forward} symbolic execution, where a symbolic engine analyzes many paths simultaneously starting its exploration from the main entry point of a program.
%its extensive usage in software testing and computer security applications\mynote{[D] this should change}, where software vulnerabilities can be found by symbolically executing programs at the level of either source or binary code. 
%A different approach is symbolic {\em backward} execution, where exploration is started from a specific point of the program (e.g., an {\tt assert} statement) and the engine proceeds backward, trying to reconstruct a valid path from an entry point of the program. Since forward symbolic execution is the mainline technique in literature, throughout this article we will always refer to this approach when using the term symbolic execution. Nonetheless, some benefits offered by symbolic backward execution will be pointed out when relevant for the discussion.

We start with a simple example that highlights many of the fundamental issues addressed in the remainder of the article.

% --------------------------------------------------------------------------------------------------------------------
\subsection{A Warm-Up Example}
\label{symbolic-execution-example}

\begin{figure}[t]
\begin{center}
\begin{tabular}{c}
\begin{lstlisting}[basicstyle=\ttfamily\scriptsize]
1.  void foobar(int a, int b) {
2.     int x = 1, y = 0;
3.     if (a != 0) {
4.        y = 3+x;
5.        if (b == 0)
6.           x = 2*(a+b);
7.     }
8.     assert(x-y != 0);
9.  }
\end{lstlisting}
\end{tabular}
\end{center}
\vspace{-2mm}
\caption{Warm-up example: which values of \texttt{a} and \texttt{b} make the \texttt{assert} fail?}
\label{fig:example-1}
%\vspace{-1.5mm}
\end{figure}

%\revedit{in the common 4-byte representation}
Consider the C code of Figure~\ref{fig:example-1} and assume that our goal is to determine which inputs make the {\tt assert} at line 8 of function \texttt{foobar} fail. Since each 4-byte input parameter can take as many as $2^{32}$ distinct integer values, the approach of running concretely function \texttt{foobar} on randomly generated inputs will unlikely pick up exactly the assert-failing inputs.
%Techniques such as random testing could generate bottomless input tests for this function. 
%However, it is unlikely that exactly the assert-failing inputs would be randomly picked up\mynote{Fuzzing?}. 
By evaluating the code using symbols for its inputs, instead of concrete values, symbolic execution overcomes this limitation and makes it possible to reason on {\em classes of inputs}, rather than single input values. 

In more detail, every value that cannot be determined by a static analysis of the code, such as an actual parameter of a function or the result of a system call that reads data from a stream, is represented by a symbol $\alpha_i$. At any time, the symbolic execution engine maintains a state $(stmt,~\sigma,~\pi)$ where:

\begin{itemize}[itemsep=2pt]

\item $stmt$ is the next statement to evaluate. For the time being, we assume that $stmt$ can be an assignment, a conditional branch, or a jump (more complex constructs such as function calls and loops will be discussed in  Section~\ref{se:path-explosion}).

%\item $\sigma$ is a {\em symbolic store} that associates program variables with expressions over \mynote{[D] $\alpha_i$ also concrete?} concrete and symbolic values $\alpha_i$.

\item $\sigma$ is a {\em symbolic store} that associates program variables with either expressions over concrete values or symbolic values $\alpha_i$.

\item $\pi$ denotes the {\em path constraints}, i.e., is a formula that expresses a set of assumptions on the symbols $\alpha_i$ due to branches taken in the execution to reach $stmt$. At the beginning of the analysis, $\pi=true$.

\end{itemize}

\noindent Depending on $stmt$, the symbolic engine changes the state as follows:

\begin{itemize}[topsep=3pt,itemsep=2pt] % TODO
  \item The evaluation of an assignment $x=e$ updates the symbolic store $\sigma$ by associating $x$ with a new symbolic expression $e_s$. We denote this association with $x\mapsto e_s$, where $e_s$ is obtained by evaluating $e$ in the context of the current execution state and  can be any expression involving unary or binary operators over symbols and concrete values.
  
%   $\alpha_i = e$: when an expression $e$ is assigned to a symbol $\alpha_i$, $pc$ is extended by adding a constraint on $\alpha_i$:
%    \[ pc \gets pc \wedge \alpha_i = e\]
%  where $e$ can be any expression, involving unary or binary operators, over symbols and constants.

  \item The evaluation of a conditional branch ${\tt if}~e~{\tt then}~s_{true}~{\tt else}~s_{false}$ affects the path constraints $\pi$. The symbolic execution is forked by creating two execution states with path constraints $\pi_{true}$ and $\pi_{false}$, respectively, which correspond to the two branches: $\pi_{true}=\pi \wedge e_s$ and $\pi_{false}=\pi \wedge \neg e_s$, where $e_s$ is a symbolic expression obtained by evaluating $e$. 
%        \[ (s_{true}, pc_{true}) \text{ where } pc_{true} = pc \wedge e \]
%        \[ (s_{false}, pc_{false}) \text{ where } pc_{false} = pc \wedge \neg e \]
    Symbolic execution independently proceeds on both states.

  \item The evaluation of a jump {\tt goto} $s$ updates the execution state by advancing the symbolic execution to statement $s$. 
\end{itemize}

%\subsection{Example}
%\label{symbolic-execution-example}

%\begin{figure}[t]
%  \centering
%  \includegraphics[width=1.0\columnwidth]{images/example} 
%  \caption{Symbolic execution tree of the function {\tt foobar}. Each execution state is labeled with an alphabet letter. Side effects on execution states are highlighted in gray. Leaves are evaluated against division by zero error. For the sake of presentation the conjunction of constraints is shown as a list of constraints. }
%  \label{fig:example-symbolic-execution}
%\end{figure}

\begin{figure}[t]
  \centering
  \includegraphics[width=0.975\columnwidth]{images/execution-tree.eps} 
  \caption{Symbolic execution tree of function {\tt foobar} given in Figure~\ref{fig:example-1}. Each execution state, labeled with an upper case letter, shows the statement to be executed, the symbolic store $\sigma$, and the path constraints $\pi$. Leaves are evaluated against the condition in the {\tt assert} statement. }
%For the sake of presentation the conjunction of constraints is shown as a list of constraints. }
  \label{fig:example-symbolic-execution}
  %\vspace{-1mm}
\end{figure}

\noindent A symbolic execution of function {\tt foobar}, which can be effectively represented as a tree, is shown in Figure~\ref{fig:example-symbolic-execution}. Initially (execution state $A$) the path constraints are {\tt true} and input arguments {\tt a} and {\tt b} are associated with symbolic values. 
After initializing local variables {\tt x} and {\tt y} at line 2, the symbolic store is updated by associating {\tt x} and {\tt y} with concrete values 1 and 0, respectively (execution state $B$). Line 3 contains a conditional branch and the execution is forked: depending on the branch taken, a different statement is evaluated next and different assumptions are made on symbol $\alpha_a$ (execution states $C$ and $D$, respectively). In the branch where $\alpha_a\neq 0$, variable {\tt y} is assigned with ${\tt x}+3$, obtaining $y\mapsto 4$ in state $E$ because $x\mapsto 1$ in state $C$. In general, arithmetic expression evaluation simply manipulates the symbolic values.
After expanding every execution state until the {\tt assert} at line 8 is reached on all branches, we can check which input values for parameters {\tt a} and {\tt b} can make the {\tt assert} fail. By analyzing execution states $\{D,G,H\}$, we can conclude that only $H$ can make {\tt x-y = 0} true. The path constraints for $H$ at this point implicitly define the set of inputs that are unsafe for {\tt foobar}. 
In particular, any input values such that:
 \[ 2(\alpha_a+\alpha_b)-4 = 0 \wedge \alpha_a \neq 0 \wedge \alpha_b = 0 \]
will make {\tt assert} fail. An instance of unsafe input parameters can be eventually determined by invoking an {\em SMT solver}~\cite{BKM14} to solve the path constraints, which in this example would yield $a = 2$ and $b = 0$. % HandbookOfSAT2009

%Notice\mynote{Say earlier?} that a constraint solver is also needed when evaluating the satisfiability of branch conditions.

% --------------------------------------------------------------------------------------------------------------------
\subsection{Challenges in Symbolic Execution}
\label{example-discussion}

In the example discussed in Section~\ref{symbolic-execution-example} symbolic execution can identify {\em all} the possible unsafe inputs that make the {\tt assert} fail. This is achieved through an exhaustive exploration of the possible execution states. From a theoretical perspective, exhaustive symbolic execution provides a {\em sound} and {\em complete} methodology for any decidable analysis. Soundness prevents false negatives, i.e., all possible unsafe inputs are guaranteed to be found, while completeness prevents false positives, i.e.,  input values deemed unsafe are actually unsafe. As we will discuss later on, exhaustive symbolic execution is unlikely to scale beyond small applications. Hence, in practice we often settle for less ambitious goals, e.g., by trading soundness for performance.

Challenges that symbolic execution has to face when processing real-world code can be significantly more complex than those illustrated in our warm-up example. Several observations and questions naturally arise:

\begin{itemize}[itemsep=1mm]
%%%
\item \noindent {\em Memory}: how does the symbolic engine handle pointers, arrays, or other complex objects? Code manipulating pointers and data structures may give rise not only to symbolic stored data, but also to addresses being described by symbolic expressions.
%Any arbitrarily complex object can be regarded as an array of bytes and each byte associated with a distinct symbol. However, when possible, exploiting structural properties of the data may be more convenient: for instance, relational bounds on the class fields in object-oriented languages could be used for refining the search performed by symbolic execution.
%%%
\item {\em Environment}: how does the engine handle interactions across the software stack? Calls to library and system code can cause side-effects, e.g., the creation of a file or a call back to user code, that could later affect the execution and must be accounted for. However, evaluating any possible interaction outcome may be unfeasible.
%: it would give rise to a large number of states, while only a fraction of them can \mynote{[D] likely?}actually happen in a non-symbolic scenario.
%%\mytempedit{Also, third-party closed-source components and popular frameworks (e.g., Java Swing and Android) pose further challenges to an executor, for instance because of the control flows occurring within them through callbacks.}\mynote{CD: may be dropped if we run out of space}
% Real-world applications constantly interact with the environment (e.g., the file system or the network) through libraries and system calls. These interactions may cause side-effects (such as the creation of a file) that could later affect the execution and must be therefore taken into account. Evaluating any possible interaction outcome is generally unfeasible: it could generate a large number of execution states, of which only a small number can actually happen in a non-symbolic scenario. %A typical strategy is to consider popular library and system routines and create models that can help the symbolic engine analyze only significant outcomes.
%%%
  \item {\em State space explosion}: how does symbolic execution deal with path explosion?
%\mynote{[D] I felt it was too long and loop-centric} 
Language constructs such as loops might exponentially increase the number of execution states. It is thus unlikely that a symbolic execution engine can exhaustively explore all the possible states within a reasonable amount of time. %In practice, heuristics are used to guide exploration and prioritize certain states first (e.g., to maximize code coverage). In addition, 
%\mytempedit{Efficient mechanisms can be implemented for preventing repeated exploration of the same piece of code
%\mytempedit{for skipping over states subsumed by previously explored paths} 
%and for evaluating multiple states in parallel without running out of resources.}
%%A loop\mynote{IF: rimuoverei la prima frase, perche' va detto?} can be encoded using conditional branches and {\tt goto} statements, which is typical  when compiling high-level languages to an intermediate representation or native code. 
%Choosing the number of loop iterations to analyze is especially critical when this number cannot be determined in advance (e.g., depends on an input parameter). The naive approach of unrolling iterations for every valid bound would result in a prohibitively large number of states. Typical solutions are to compute an underapproximation of the analysis by limiting the number of iterations to some value $k$, thus trading speed for soundness. Other approaches infer loop invariants through static analysis  and use them to merge equivalent states. % \mynote{i.e. or e.g.?}  (e.g., when differences are not observable from outside the loop body). 
  %In practice, several heuristics must be exploited to prioritize evaluation of some states, hoping to still be able to spot interesting things. Moreover, the symbolic execution engine should include efficient mechanism for efficiently evaluating in parallel different execution states without running out of computational resources.
%%%
  \item {\em Constraint solving}: what can a constraint solver do in practice?
  %{\em What is a constraint solver in practice}? \\
SMT solvers can scale to complex combinations of constraints over hundreds of variables. However, constructs such as non-linear arithmetic pose a major obstacle to efficiency.
%Constraint solvers suffer from a number of limitations. They can typically handle complex constraints in a reasonable amount of time only if they are made of linear expressions over their constituents.
%Constraint solvers suffer from a number of limitations. They can typically handle complex constraints in a reasonable amount of time only if they are made of linear expressions over their constituents. %Symbolic execution engines normally implement a number of optimizations to make queries as much {\em solver-friendly} as possible, for instance by splitting queries into independent components to be processed separately or by performing algebraic simplifications.
%%%
\iffullver{  \item {\em Binary code}: what issues can arise when symbolically executing binary code?
  %what are the disadvantages of symbolically executing binary code?
 While the warm-up example of Section~\ref{symbolic-execution-example} is written in C, in several scenarios binary code is the only available representation of a program. However, having the source code of an application can make symbolic execution significantly easier, as it can exploit high-level properties (e.g., object shapes) that can be inferred statically by analyzing the source code.
 }{}
%(e.g., the maximum size of a buffer or the number of iterations for a loop).
%%%   
\end{itemize}
%Depending on the specific application context of symbolic execution

\noindent Depending on the specific context in which symbolic execution is used, different choices and assumptions are made to address the questions highlighted above. Although these choices typically affect soundness or completeness, in several scenarios a partial exploration of the space of possible execution states may be sufficient to achieve the goal (e.g., identifying a crashing input for an application) within a limited time budget.


%different choices and assumptions are made to address the above questions. Although soundness and completeness of symbolic execution may be negatively affected by these choices, there are several application scenarios where a partial exploration of the possible execution states is sufficient for reaching the ultimate goal (e.g., identify a single input that crashes an application).

% --------------------------------------------------------------------------------------------------------------------
\subsection{Related Work}
\label{ss:related-surveys}

Symbolic execution has been the focus of a vast body of literature. As of August 2017, Google Scholar reports 742 articles that include the exact phrase ``symbolic execution'' in the title. Prior to this survey, other authors have contributed technical overviews of the field, such as \cite{PV-JSTTT09} and \cite{CS-CACM13}. \cite{CHEN20131758} focuses on the more specific setting of automated test generation: it provides a comprehensive view of the literature, covering in depth a variety of techniques and complementing the technical discussions with a number of running examples.
%Besides  complementing the technical discussions with a number of running examples, it covers in depth recent techniques for key aspects such as memory modelling, environment interaction, path explosion, and constraint solving.

% --------------------------------------------------------------------------------------------------------------------
\subsection{Organization of the Article}
\label{ss:article-organization}

The remainder of this article is organized as follows. In Section~\ref{se:executors} we discuss the overall principles and evaluation strategies of a symbolic execution engine. Section~\ref{memory-model} through Section~\ref{se:constraint-solving} address the key challenges that we listed in Section~\ref{example-discussion}, while Section~\ref{se:hang} discusses how recent advances in other areas could be applied to enhance symbolic execution techniques. Concluding remarks are addressed in Section~\ref{se:conclusions}.
\myinput{executors}
\myinput{memory}
\myinput{environment}
\myinput{explosion}
\myinput{constraints}
% !TEX root = main.tex

\section{Further Directions}
\label{se:hang}

In this section we discuss how recent advances in related research areas could be applied or provide potential directions to enhance the state of the art of symbolic execution techniques. In particular, we discuss separation logic for data structures, techniques from the program verification and program analysis domains for dealing with path explosion, and symbolic computation for dealing with non-linear constraints.
% , i.e., polynomial constraints over variables

% Reynolds02,IO-POPL01
\subsection{Separation Logic}
%\mytempedit{
Checking memory safety properties for pointer programs is a major challenge in program verification. Recent years have witnessed {\em separation logic} (SL)~\cite{Reynolds02} emerging as one leading approach to reason about heap manipulations in imperative programs. SL extends Hoare logic to facilitate reasoning about programs that manipulate pointer data structures, and allows expressing complex invariants of heap configurations in a succinct manner.

At its core, a {\em separating conjunction} binary operator $*$ is used to assert that the heap can be partitioned in two components where its arguments separately hold. For instance, predicate $A * x\mapsto�[n:y]$ says that there is a single heap cell $x$ pointing to a record that holds $y$ in its $n$ field, while $A$ holds for the rest of the heap.

Program state is modeled as a {\em symbolic heap} $\Pi\,\brokenvert\,\Sigma$: $\Pi$ is a finite set of pure predicates related to variables, while $\Sigma$ is a finite set of heap predicates. Symbolic heaps are SL formulas that are symbolically executed according to the program's code using an abstract semantics. SL rules are typically employed to support entailment of symbolic heaps, to infer which heap portions are not affected by a statement, and to ensure termination of symbolic execution via abstraction (e.g., using a widening operator).

A key to the success of SL lies in the local form of reasoning enabled by its $*$ operator, as it allows specifications that speak about the sole memory accessed by the code. This also fits together with the goal of deriving inductive definitions to describe mutable data structures. When compared to other verification approaches, the annotation burden on the user is rather little or often absent. For instance, the shape analysis presented in~\cite{CDO-JACM11} uses bi-abduction to automatically discover invariants on data structures and compute composable procedure summaries in SL.

% verification (Section~\ref{se:constraint-solving})
Several tools based on SL are available to date, for instance, for automatic memory bug discovery in user and system code, and verification of annotated programs against memory safety properties or design patterns. While some of them implement tailor-made decision procedures, \cite{BPS-ENTCS09,PWZ-CAV13} have shown that provers for decidable SL fragments can be integrated in an SMT solver, allowing for complete combinations with other theories relevant to program verification. This can pave the way to applications of SL in a broader setting: for instance, a symbolic executor could use it to reason inductively about code that manipulates structures such as lists and trees. While symbolic execution is at the core of SL, to the best of our knowledge there have not been uses of SL in symbolic executors to date.%We believe this might represent a promising research direction to follow.
%}

\subsection{Invariants} 
\label{ss:invariants}
Invariants are crucial for verifiers that can prove programs correct against their full functional specification. An invariant is a predicate true for an initial state and for each state reachable from it. Leveraging invariants can be beneficial to symbolic executors, in order to compactly capture the effects of a loop and reason about them. Unfortunately, we are not aware of symbolic executors taking advantage of this approach. One of the reasons might lie in the difficulty of computing loop invariants without requiring manual intervention from domain experts. In fact, lessons from the verification practice suggest that providing loop invariants is much harder compared to other specification elements such as method pre/post-conditions.

% a number of works [...] and might be
However, many researchers have recently explored techniques for inferring loop invariants automatically or with little human help~\cite{FMV-CSUR14}, which might be of interest for the symbolic execution community for a more efficient handling of loops. These approaches normally target inductive predicates, which are closed under the state transition relation (i.e., they make no reference to past behavior). Notice that all inductive predicates are invariants, but the converse is not true.

% that rank all -> over all
{\em Termination analysis} has been applied to verify program termination for industrial code: a formal argument is typically built by using one or more ranking functions over all the possible states in the program such that for every state transition, at least one function decreases~\cite{CPR-PLDI06}. Ranking functions can be constructed in a number of ways, e.g., by lazily building an invariant using counterexamples from traversed loop paths~\cite{GMR-PLDI15}. A termination argument can also be built by reasoning over transformed programs where loops are replaced with summaries based on transition invariants~\cite{TSW-TACAS11}. It has been observed that most loops in practice have relatively simple termination arguments~\cite{TSW-TACAS11}: the discovered invariants may thus not be rich enough for a verification setting~\cite{GFM-TSE15}. However, a constant or parametric bound on the number of iterations may still be computed from a ranking function and an invariant~\cite{GMR-PLDI15}.

{\em Predicate abstraction} is a form of abstract interpretation over a domain constructed using a given set of predicates, and has been used to infer universally quantified loop invariants~\cite{FQ-POPL02}, which are useful when manipulating arrays. Predicates can be heuristically collected from the code or supplied by the user: it would be interesting to explore a mutual reinforcing combination with symbolic execution, with additional useful predicates being originated during the symbolic exploration.

{\em LoopFrog}~\cite{LOOPFROG-ATVA08} replaces loops using a symbolic abstract transformer with respect to a set of abstract domains, obtaining a conservative abstraction of the original code. Abstract transformers are computed starting from the innermost loop, and the output is a loop-free summary of the program that can be handed to a model checker for verification. This approach can also be applied to non-recursive function calls, and might deserve some investigation in symbolic executors. 

Loop invariants can also be extracted using {\em interpolation}, a general technique that has already been applied in symbolic execution for different goals (Section~\ref{ss:interpolation}). %More generally, we believe that modern advances in invariant generation can provide potential solutions for handling loops more efficiently in a symbolic executor.
%
%}

% DROPPED as it's an application
\iffullver{
On the other hand, symbolic execution has proved useful to derive loop invariants. For instance, if a program contains an assertion after the loop, the approach presented in~\cite{PV-SPIN04} works backwards from the property to be checked and it iteratively applies approximation to derive loop invariants. The main idea is to pick the asserted property as the initial invariant candidate and then to exploit symbolic execution to check whether this property is inductive. If the invariant cannot be verified for some loop paths, it is replaced by a different invariant. The next candidate for the invariant is generated by exploiting the path constraints for the paths on which the verification has failed. Additional refinements steps are performed to guarantee termination. % [D] say weakness in not being able to find invariants that do not directly depends on the path conditions?
}{}

%Nevertheless, even symbolic execution can be used to derive loop invariants. Indeed, if a program contains an assertion after the loop, the approach presented in~\cite{PV-SPIN04} works backwards from the property to be checked and it iteratively applies approximation to derive loop invariants. The main idea is to pick the asserted property as the initial invariant candidate and then to exploit symbolic execution to check whether this property is inductive. If the invariant cannot be verified for some loop paths, it is replaced by a different invariant. The next candidate for the invariant is generated by exploiting the path constraints for the paths on which the verification has failed. Additional refinements steps are performed to guarantee termination.
%this can be exploited by a symbolic engine for automatically discovering some invariants over the loop. In~\cite{PV-SPIN04}, this is achieved by iteratively using \mynote{[D] Define?} invariant strengthening and approximation techniques. 

\subsection{Function Summaries}
%\mytempedit{
Function summaries (Section~\ref{ss:summarization}) have largely been employed in static and dynamic program analysis, especially in program verification. A number of such works could offer interesting opportunities to advance the state of the art in symbolic execution. For instance, the Calysto static checker~\cite{CALYSTO-ICSE08} walks the call graph of a program to construct a symbolic representation of the effects of each function, i.e., return values, writes to global variables, and memory locations accessed depending on its arguments. Each function is processed once, possibly inlining effects of small ones at their call sites. Static checkers such as Calysto and Saturn~\cite{SATURN-POPL05} trade scalability for soundness in summary construction, as they unroll loops only to a small number of iterations: their use in a symbolic execution setting may thus result in a loss of soundness. More fine-grained summaries are constructed in~\cite{RACERX-SOSP03} by taking into account different input conditions using a summary cache for memoizing the effects of a function.

\cite{SFS11} proposes a technique to extract function summaries for model checking where multiple specifications are typically checked one a time, so that summaries can be reused across verification runs. In particular, they are computed as over-approximations using interpolation (Section~\ref{ss:interpolation}) and refined across runs when too weak. The strength of this technique lies in the fact that an interpolant-based summary can capture all the possible execution traces through a function in a more compact way than the function itself. The technique has later been extended to deal with nested function calls in~\cite{SFS12}.%, which discusses an useful application in incremental update checking of programs.
%}

\subsection{Program Analysis and Optimization}
%\mytempedit{
We believe that the symbolic execution practice might further benefit from solutions that have been proposed for related problems in the programming languages realm. For instance, in the parallel computing community transformations such as {\em loop coalescing}~\cite{BGS-CSUR94} can restructure nested loops into a single loop by flattening the iteration space of their indexes. Such a transformation could potentially simplify a symbolic exploration, empowering search heuristics and state merging strategies. 

{\em Loop unfolding}~\cite{SK-SIGPLAN-NOTICES04} may possibly be interesting as well, as it allows exposing ``well-structured'' loops (e.g., showing invariant code, or having constants or affine functions as subscripts of array references) by peeling several iterations.

{\em Program synthesis} automatically constructs a program satisfying a high-level specification~\cite{PR-POPL89}. The technique has caught the attention of the verification community since~\cite{Solar-Lezama08} has shown how to find programs as a solution to SAT problems.
In Section~\ref{se:environment-thirdparty} we discussed its usage in~\cite{JQF-ICSE16} to produce compact models for complex Java frameworks: the technique takes as inputs classes, methods and types from a framework, along with tutorial programs (typically those provided by the vendor) that exercise its parts. We believe this approach deserves further investigation in the context of the path explosion problem. It could potentially be applied to software modules such as standard libraries to produce concise models that allow for a more scalable exploration of the search space, as synthesis can capture an external behavior while abstracting away entanglements of the implementation. %
%has shown that program synthesis can be used to build models for complex Java components by abstracting away the details and entanglements of their implementations while capturing their functional behavior. In particular, 

\subsection{Symbolic Computation}
%In particular, studies in the area of {\em symbolic computation}, also known as computer algebra,
Although the satisfiability problem is known to be NP-hard already for SAT, the mathematical developments over the past decades have produced several practically applicable methods to solve arithmetic formulas. In particular, advances in {\em symbolic computation} have produced powerful methods such as Gr\"{o}bner bases for solving systems of polynomial constraints, cylindrical algebraic decomposition for real algebraic geometry, and virtual substitution for polynomial real arithmetic formulas in which the degree of polynomials is no more than four~\cite{Abraham15}.

% handling complex Boolean constraints and; quantifier-free non-linear
% have proven to be
While SMT solvers are very efficient at combining theories and heuristics when processing complex expressions, they make use of symbolic computation techniques only to a little extent, and their support for non-linear real and integer arithmetic is still in its infancy~\cite{Abraham15}. To the best of our knowledge, only Z3~\cite{Z3-TACS08} and SMT-RAT~\cite{SMTRAT15} can reason about them both.
%provide support to reason about them both.

\cite{Abraham15} states that using symbolic computation techniques as theory plugins for SMT solvers is a promising symbiosis, as they provide powerful procedures for solving conjunctions of arithmetic constraints. The realization of this idea is hindered by the fact that available implementations of such procedures do not comply with the incremental, backtracking and explanation of inconsistencies properties expected of SMT-compliant theory solvers. One interesting project to look at is SC\textsuperscript{2}~\cite{SC2}, whose goal is to create a new community aiming at bridging the gap between symbolic computation and satisfiability checking, combining the strengths of both worlds in order to pursue problems currently beyond their individual reach.

Further opportunities to increase efficiency when tackling non-linear expressions might be found in the recent advances in {\em symbolic-numeric computation}~\cite{HandbookOfCompAlgebra}. In particular, these techniques aim at developing efficient polynomial solvers by combining numerical algorithms, which are very efficient in approximating local solutions but lack a global view, with the guarantees from symbolic computation techniques. This hybrid techniques can extend the domain of efficiently solvable problems, and thus be of interest for non-linear constraints from symbolic execution.

% !TEX root = main.tex

\section{Conclusions}
\label{se:conclusions}

Symbolic execution techniques have evolved significantly in the last decade, with notable applications to compelling problems from several domains like software testing (e.g., test input generation, regression testing), security (e.g., exploit generation, authentication bypass), and code analysis (e.g., program deobfuscation, dynamic software updating). This trend has not only improved existing solutions, but also led to novel ideas and, in some cases, to major practical breakthroughs. For instance, the push for scalable automated program analyses in security has culminated in the 2016 DARPA Cyber Grand Challenge, which hosted systems for detecting and fixing vulnerabilities in unknown software with no human intervention, such as {\sc Angr}~\cite{ANGR-SSP16} and {\sc Mayhem}~\cite{MAYHEM-SP12}, that competed for nearly \$4M in prize money.

%\noindent
This survey has discussed some of the key aspects and challenges of symbolic execution, presenting for a broad audience the basic design principles of symbolic executors and the main optimization techniques. We hope it will help non-experts grasp the key inventions in this exciting line of research, inspiring further work and new ideas.

\specialcomment{online}{
\begingroup
\subsection*{ELECTRONIC APPENDIX}
\phantomsection\addcontentsline{toc}{subsection}{Electronic Appendix}
}{%
\endgroup
}

%\begin{online}
\subsection*{ELECTRONIC APPENDIX}
The online appendix of this manuscript discusses a selection of prominent applications of symbolic execution techniques, addresses further challenges that arise in the analysis of programs in binary form, and provides a list of popular symbolic engines.
%\end{online}

\begin{acks}
We thank the anonymous referees for their valuable comments and helpful suggestions. This work is supported in part by a grant of the Italian Presidency of the Council of Ministers and by the CINI (Consorzio Interuniversitario Nazionale Informatica) National Laboratory of Cyber Security. % (Consorzio Interuniversitario Nazionale Informatica) 
\end{acks}

\iffalse
Techniques for symbolic execution have evolved significantly in the last decade, leading to major practical breakthroughs. In 2016, the DARPA Cyber Grand Challenge hosted systems that can detect and fix vulnerabilities in unknown software with no human intervention, such as {\sc Angr}~\cite{ANGR-SSP16} and {\sc Mayhem}~\cite{MAYHEM-SP12}, which won the \$2M first prize. {\sc Mayhem} was also the first autonomous software to play the Capture-The-Flag contest at the DEF CON 24 hacker convention\footnote{\url{https://www.defcon.org/html/defcon-24/dc-24-ctf.html}.}. The event demonstrated that tools for automatic exploit detection based on symbolic execution can be competitive with human experts, paving the road to unprecedented applications %and the rise of start-ups 
that have the potential to shape software %security and 
reliability in the next decades. 

This survey has discussed some of the key aspects and challenges of symbolic execution, presenting them for a broad audience. 
To explain the basic design principles of symbolic executors and the main optimization techniques, we have focused on single-threaded applications with integer arithmetic. Symbolic execution of multi-threaded programs is treated, e.g., \iffullver{in~\cite{KPV-TACAS03,SA-HVC06,CLOUD9-EUROSYS11,FHR-ESEC13,BGC-OOPSLA14,GKW-ESEC15}}
{in~\cite{BGC-OOPSLA14,GKW-ESEC15}}, 
%{in~\cite{FHR-ESEC13,BGC-OOPSLA14,GKW-ESEC15}}, 
while techniques for programs that manipulate floating point data are addressed \iffullver{in, e.g., \cite{M-STVR01,BGM-STVR06,LTH-ICTSS10,CCK-EUROSYS11,BVL-POPL13,CCK-TSE14,RPW-SIGSOFT15}}
{in, e.g., \cite{RPW-SIGSOFT15}}.
%{in, e.g., \cite{BVL-POPL13,CCK-TSE14,RPW-SIGSOFT15}}.

We hope that this survey will help non-experts grasp the key inventions in the exciting line of research of symbolic execution, inspiring further work and new ideas.
\fi


%\myparagraph{Acknowledgements}
%This work is partially supported by a grant of the Italian Presidency of Ministry Council and by the CINI  (Consorzio Interuniversitario Nazionale Informatica) Cybersecurity National Laboratory.
%This work is supported in part by a grant of the Italian Presidency of the Council of Ministers and by the CINI (Consorzio Interuniversitario Nazionale Informatica) National Laboratory of Cyber Security.

\ifdefined\arxivver
\myparagraph{Live Version of this Article}
We complement the traditional scholarly publication model by maintaining a live version of this article at {\href{https://github.com/season-lab/survey-symbolic-execution}{https://github.com/season-lab/survey-symbolic-execution/}}. The live version incorporates continuous feedback by the community, providing post-publication fixes, improvements, and extensions.
\fi


% Bibliography
%\bibliographystyle{abstract} 
\bibliographystyle{ACM-Reference-Format-Journals}
\bibliography{symbolic}

% History dates
%\received{--- 2016}{--- XXXX}{---- XXXX}

\end{document}

% End of v2-acmsmall-sample.tex (March 2012) - Gerry Murray, ACM


